% (c) 2009-2021 by Markus Leupold-Löwenthal
% This file is released under CC BY-SA 4.0. Please do not apply one-way compatible licenses.
% Original version by André Frenzer.

\renewcommand{\lastroadVersion}{v1.0.1}

% CHANGELOG-de
%
% 1.0.1
%   - Layout Credits
% 1.0
%   - Erstveröffentlichung
% 0.2
%   - Tippfehler
% 0.1
%   - Erstfassung

% --- language dependent typography stuff --------------------------------------

\renewcommand{\fsNormal}{\fontsize{9pt}{11.25pt plus 0.1pt minus 0pt}}
\renewcommand{\fsSmall}{\fontsize{8.5pt}{9.5pt plus 0.1pt minus 0pt}}

% --- pdf metadata -------------------------------------------------------------

\hypersetup{
	pdfsubject={Ein kurzes NIP'AJIN Szenario in einer Wüsten-Endzeit in den USA.},
}

% --- fine print ---------------------------------------------------------------

\renewcommand{\lastroadCredits}{
	Text:~André Frenzer; Lektorat:~Onno Tasler, Markus Leupold-Löwenthal%
}

% --- main texts ---------------------------------------------------------------

\renewcommand{\lastroadPlayers}{%
	Ein Szenario von André~Frenzer für drei bis vier Charaktere.
}
\renewcommand{\lastroadHeadline}{Der letzte Weg}
\renewcommand{\lastroadToc}{Szenario: Der letzte Weg}
\renewcommand{\lastroadText}{%
	{\itshape
		Alles begann vor elf Monaten mit der Entrückung. Von einem Tag auf den anderen verschwanden jene Menschen, die von ihren Göttern als ehrfürchtig und gerecht empfunden wurden. Während die über Nacht arg reduzierten Nachrichtenagenturen heiß liefen und immer mehr Berichte von verschwundenen Menschen über den Äther liefen, stellten die Verbliebenen fest, dass sie sich den Planeten nur mehr mit dem Abschaum der Menschheit teilten: Gewalttätige, Mörder, Kriegsverbrecher.

		Noch während die Übriggebliebenen mit dem Recht des Stärkeren darum kämpften, eine neue Weltordnung zu etablieren, erschallte die erste Posaune der Götter. Gerade verklungen -- die Menschen hatten kaum gewagt, ihre zitternden Häupter wieder zu erheben -- erschien Seth, der alte Gott des Chaos, der Gewalt und des Verderbens. Mit heiligem Eifer wandelte er über die Erde, Tod und Vernichtung über jene bringend, die ihm in den Weg kamen. Er ebnete den Weg für Horus, Re und Isis, die allesamt das Angesicht der Welt neu formten und in eine gigantische Wüste verwandelten.

		Schließlich kam Am-Hehu, der \say{Verschlinger der Millionen}, und hielt blutige Ernte unter den wenigen, die noch übrig waren. In seinem Gefolge wandelten die Ushebti, gigantische tierköpfige Humanoide, die blieben, als die Götter wieder verschwanden.

		Seit einem Monat hat sich der Staub gelegt, und die wenigen Überlebenden kämpfen in den Wüsten dieser neuen Welt um Nahrung, Rohstoffe, Waffen und ihre Menschlichkeit.
	}

	\mysection{Setting}

		\noindent
		Dieses Szenario spielt in einer Wüste, die noch vor wenigen Wochen das dicht besiedelte und fruchtbare Neuengland war. Das Wandeln der Götter bereitete jedoch dem satten Grün ein Ende und übrig blieb nur der Wüstensand. Die Charaktere wandern nach Osten zur Küste, genauer gesagt Richtung Boston. Auf der Suche nach Nahrung und Überlebenden stoßen sie auf ein ehemaliges Gefängnis.

		\subsection{Das Gefängnis}

		Die Mauern und Zäune rund um das Gelände sind großteils eingerissen und auch die Wände der einzelnen Bauteile sind teils eingestürzt. Auf dem einzigen verbliebenen Wachturm, der trotzig in den Himmel ragt, weht eine Fahne mit dem altägyptischen Ankh. Das Gefängnis besteht aus insgesamt drei Trakten.

		Der \keyword{Westtrakt} hat drei baugleiche Etagen. Ein langer Korridor zieht sich von Nord nach Süd durch das gesamte Gebäude. Links und rechts des Korridors befinden sich vergitterte Gefängniszellen. Der gesamte Zellentrakt ist bar menschlichen Lebens, allerdings können die Charaktere kleinere Ausrüstungsgegenstände wie Zigaretten, Rasierer oder Plastikteller finden, sofern sie dafür Verwendung haben. Ein aufgebrochener Medizinschrank enthält noch genug Verbandsmaterial für fünf Heilversuche.

		Der \keyword{Osttrakt} liegt unter dem Schatten des mit der Ankh-Flagge geschmückten Wachturmes. Er entspricht im Aufbau dem Westtrakts. Hier haben sich in der oberen Etage überlebende \keyword{Häftlinge} zusammengerottet (\HD4, \AD6, \RD6, Schießen-1; Gefängniskleidung; Anzahl: wie SC). Sie haben sich der Waffen der verschwundenen oder toten Sicherheitsbeamten bemächtigt und schießen, ohne Fragen zu stellen. Können die Charaktere die Gestalten trotzdem in ein Gespräch verwickeln, so erklären diese, dass sie unter dem Schutz des ägyptischen Ankh eine neue Weltordnung etablieren wollen und auf der Suche nach Gleichgesinnten sind. Sie sind hoffnungslos wahnsinnig und können nicht dazu überredet werden, die Charaktere zu begleiten.

		Der nur zweigeschossige \keyword{Nordtrakt} liegt gedrungen und finster zwischen den anderen beiden und enthält die ehemalige Verwaltung. Neben der Wäscherei und den Aufenthalts- und Pausenräumen des Personals finden sich hier das Büro des Direktors sowie die Gefängniskantine. In der Kantine lässt sich ein großzügiger Vorrat Konservendosen auftreiben sowie einige Küchenmesser. In den Spinden der Gefängniswärter im Pausenraum liegen einige Schlagstöcke und Elektroschocker. Der Direktor sitzt als mumifizierter Leichnam über seinen Schreibtisch gebeugt in seinem Büro. Kommen ihm die Charaktere zu nahe, erwacht er zu unheiligem Leben und greift an. \keyword{Mumifizierter Direktor}: \HD8, \AD8, \RD6, Verteidigung+1, brennbar (doppelter Schaden).

	\mysection{Charaktere}

		\noindent
		In diesem Szenario verkörpern die Spieler die härtesten Schläger aus dem ohnehin reichlich brutalen Abschaum der Menschheit, z. B.:

		\keyword{Butcher:} Großer, muskulöser, glatzköpfiger Stier\-nacken. \HD10, Stark+2, Schlau-1, Geschick-1.

		\keyword{Clarke:} Unrasierter Revolverheld mit nervösem Finger. \HD8, Heimlichkeit+2, Nervös-1, Sinne-1.

		\keyword{Betty:} Fettleibige Drogendealerin mit unflätiger Ausdrucksweise. \HD8, Wissen+2, Soziales-1, Sportlichkeit-1.

		\keyword{Gringo:} Schweigsamer Hutträger mit zahlreichen Messern. \HD6, Flink+2, Kraft-1, Hinkebein-1.

		Jeder Charakter erhält von seinem Spieler einen besonderen Gegenstand, \zB\ eine Waffe mit doppeltem Schaden oder ein Utensil, das +1 auf bestimmte Proben gibt.

	\mysection{Ablauf}

		\noindent
		Auf ihrem Weg nach Osten stößt die Gruppe auf die Ruinen des Hochsicherheitsgefängnisses. Es sind von Außen keine Anzeichen von Bewohnern zu erkennen, so dass es lohnenswert erscheint, den Vorratsräumen einen Besuch abzustatten, um die eigenen schwindenden Vorräte aufzustocken. Es steht den Charakteren frei, in welcher Reihenfolge sie die Gebäudeteile untersuchen.

		Wenn die Charaktere das Gefängnis verlassen wollen, erwartet sie im Hof eine unschöne Überraschung. Ein \keyword{Ushebti}, eine der gigantischen humanoiden Monstrositäten, die im Gefolge der Götter stehen, hat in der Zwischenzeit seinen Weg in den Hof gefunden und untersucht den Schutt. Dieses Exemplar ist rund vier Meter groß, hat einen Schakalkopf, trägt eine goldene Plattenrüstung und einen beeindruckend großen Speer bei sich. Noch hat das Wesen die Charaktere nicht bemerkt, so dass sie für den unausweichlichen Kampf günstige Positionen einnehmen und den Ushebti beispielsweise einkreisen können. Hat die Gruppe die Überlebenden im Ostflügel unversehrt gelassen -- oder vielleicht noch gar nicht bemerkt -- werden diese auf Seiten des Ushebti in den Kampf eingreifen, da sie in ihm ein Zeichen ihrer neuen Götter sehen. \keyword{Ushebti:} \HD10, \AD10, \RD8, Verteidigung+1, Speer: 2 Schaden.

	\mysection{Ende gut, alles gut?}

		\noindent
		Hat die Gruppe den Kampf gegen den Ushebti überstanden, können sie sich mit ihren neugewonnen Vorräten auf den weiteren Weg nach Osten machen. Welche Weltuntergangskulte und weitere Gefahren noch auf ihrem Weg lauern und was sie in den Ruinen Bostons vorfinden, ist der Fantasie des Spielleiters überlassen.
}
