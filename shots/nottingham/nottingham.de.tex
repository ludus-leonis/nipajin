% (c) 2009-2021 by Markus Leupold-Löwenthal
% This file is released under CC BY-SA 4.0. Please do not apply one-way compatible licenses.

\renewcommand{\nottinghamVersion}{v1.0.3}

% CHANGELOG-de
%
% 1.0.3
%   - Malus für Robin auf Schießen
%   - kleines Lektorat
% 1.0.2
%   - 2. Lektorat Onno
% 1.0.1
%   - Lektorat Onno
% 1.0
%   - Erstfassung

% --- language dependent typography stuff --------------------------------------

\renewcommand{\fsNormal}{\fontsize{9pt}{11.25pt plus 0.1pt minus 0pt}}
\renewcommand{\fsSmall}{\fontsize{8.5pt}{9.5pt plus 0.1pt minus 0pt}}

% --- pdf metadata -------------------------------------------------------------

\hypersetup{
	pdfsubject={Ein kurzes NIP'AJIN Szenario in Nottingham Shire.},
}

% --- fine print ---------------------------------------------------------------

\renewcommand{\nottinghamCredits}{
	Text:~Markus Leupold-Löwenthal; Lektorat:~Onno Tasler%
}

% --- main texts ---------------------------------------------------------------

\renewcommand{\nottinghamPlayers}{%
	Ein Szenario für drei bis fünf Gesetzlose.
}
\renewcommand{\nottinghamHeadline}{Turnier in Nottingham}
\renewcommand{\nottinghamToc}{Szenario: Turnier in Nottingham}
\renewcommand{\nottinghamText}{%
	{\itshape
		England 1186. König Heinrich II liegt im Streit mit seinem Sohn Richard. Der Sheriff von Notthingham nutzt diese Schwäche aus und unterdrückt mithilfe der Kirche die Bauern in seinem Shire. Auch ihr fielt seiner Willkür zum Opfer und musstet in den Sherwood Forest fliehen, wo ihr euch Robin Hood und seinen Merry Men angeschlossen habt.

		Morgen soll in Nottingham das jährliche Bogenturnier stattfinden, dessen Gewinner der Sheriff einen Wunsch erfüllen will. Robin möchte dies nutzen, um den Sheriff um die Hand Lady Marians zu bitten. Doch er hat sich am Arm verletzt und schießt nicht so genau wie üblich. Es braucht ein kleines Wunder, damit er trotzdem eine Chance hat.

		Dieses Wunder trifft in Gestalt Alexander von Nackams ein, der als Irrlehrer verfolgt in den Wald flüchtete. Er schlägt vor, die Zielscheiben mit Magneten zu versehen und die traditionell beim Turnier benutzten Silberpfeile für Robin mit Eisenspitzen zu manipulieren, damit dieser trotz Verletzung sicher trifft.

		Freiwillige für die Aktion sind rasch gefunden -- ihr! Im Schutz der Dunkelheit begebt ihr euch nach Nottinghamn. Bringt die Magneten in den Zielscheiben an und besorgt ein Muster der Silberpfeile, die in der Kirche St. Nicholas geweiht werden, um eiserne Kopien herzustellen.
	}

	\mysection{Setting}

		\noindent
		Das Szenario spielt südlich des Sherwood Forests in Nottigham, das 1500 Einwohner zählt. Im Westen der Stadt schmiegt sich ein etwa 30 Meter hoher Fels in eine Biegung des Flusses Leen, auf dem die Burg von Nottingham steht. An ihrem Fuße befindet sich eine große, von einem Holzzaun umgebene Wiese. Für das Turnier wurden am Areal Tribünen und allerlei Stände aufgebaut. Direkt vor dem Zaun liegt das Normannische Viertel, in dem die schlichte Steinkirche St. Nicholas steht.

	\mysection{Charaktere}

		\noindent
		Die Spieler verkörpern Charaktere mit mittelalterlichen Berufen, \zB~Bäcker, Bauer, Färber, Kerzenzieher, Metzger, Seiler, Schneider, Steinmetz oder Weber. Aktionen, die zum Beruf passen, sind um +1 erleichtert. Sie tragen drei faustgroße, magnetische Brocken mit sich.

	\mysection{Ablauf}

		\noindent
		Szene I und II finden in der Nacht statt und können in beliebiger Reihenfolge angegangen werden. Szene III beginnt am Folgetag zu Mittag. Szene IV ist optional und vom Ausgang des Turniers abhängig.

		Im Szenario könnten es die Charaktere mehrfach mit Wachen des Sheriffs zu tun bekommen.

		\keyword{Wachen}: \HD3, \AD6, \RD4, Anzahl SC-1.

		\subsection{Szene I: Zielscheiben}

		Das Holztor zum Turnierplatz ist in der Nacht abgesperrt. Wegen des bevorstehenden Turniers wird der Zaun bewacht.

		Fünf Zielscheiben sind bei der Burgmauer aufgestellt. Um zu ihnen zu gelangen, muss der Zaun überwunden werden. Die Charaktere könnten dazu \zB~für Ablenkung sorgen, mit einer List die Wachen dazu bringen, sie in das Areal zu lassen, oder diese einfach niederringen. Sie sind dabei in guter Gesellschaft, da auch Kinder versuchen, als Mutprobe auf den Turnierplatz zu schleichen.

		Bei den mannshohen Strohzielscheiben angekommen, kann versucht werden, die Magnete in diese einzuarbeiten. Dabei können die Charaktere mehrere Brocken in einzelne Ziele einarbeiten, oder die Magnete zerschlagen, um alle Ziele zu manipulieren. Es ist rasch, aber sorgfältig vorzugehen (langfristige Aktion, \TN10 pro Zielscheibe, 1 Minute/Runde, -2 wegen Dunkelheit). Am Ende jeder zweiten Runde taucht eine Komplikation auf: neugierige Kinder, laute Ziegen, ein betrunkener Arbeiter oder ein hinter Strohballen beschäftigtes Liebespaar. Wird eine Komplikation nicht abgewandt (\TN4), halten Wachen Nachschau und müssen ruhiggestellt werden.

		\subsection{Szene II: Pfeile}

		Die Silberpfeile des Sheriffs werden über Nacht in der Kirche St. Nicholas gesegnet und dort am Altar in einer Schatulle aufbewahrt. Zum Schutz von Kirche und Pfeilen patrouilliert ein Trupp Wachen rund um das Steingebäude. Sie sind angewiesen, alle Besucher genau zu untersuchen. Vorzugeben, lediglich beten zu wollen, erleichtert das Eindringen (+1).

		Im Inneren der Kirche beten Priester die ganze Nacht über. Zum Altar zu gelangen ist nicht schwierig, doch die Priester werden nicht zulassen, dass Besucher die Schatulle öffnen oder gar einen Pfeil entnehmen. Notfalls rufen sie die Wachen.

		Es gilt als Sünde, in der Kirche zu fluchen, zu stehlen oder Gewalt auszuüben. Sollten die Charaktere es schaffen, hier nicht zu sündigen, wird der Herr wohlwollend auf sie herabsehen. Es ist auch gar nicht nötig, einen Pfeil zu stehlen, es reicht, einen Abdruck oder eine Skizze anzufertigen.

		Lassen die Charaktere alle Pfeile mitgehen, wird das Turnier abgesagt und das Szenario endet vorzeitig.

		\subsection{Szene III: Turnier}

		Mit Hörnerklang startet am nächsten Tag das Turnier. Acht Schützen nehmen teil, darunter Robin Hood, der sich mit einem falschen Bart tarnt und als Monsieur Chapeau ausgibt. Der Sheriff betrachtet das Schauspiel von einer Tribüne, Marian sieht aus der Ferne vom Fenster ihres Turmzimmers zu.

		Das Turnier dauert drei Runden. Die Schützen erhalten zu Beginn jeder Runde einen Pfeil und eine zufällige Zielscheibe zugeordnet. Robin muss jede Runde versuchen, den echten Pfeil durch eine Kopie zu ersetzen (\TN4) und ggf. zu einer manipulierten Zielscheibe zu wechseln (\TN4). Ablenkungsmanöver der Charaktere erleichtern ihm das (+1). Letztlich schießt Robin: Er erhält +1 göttlichen Beistand, wenn in der Kirche nicht gesündigt wurde. Bei Einsatz eines Eisenpfeils erhält er zusätzlich +2 pro ganzem Magnet / +1 pro Magnetfragment in seiner Zielscheibe. \keyword{Robin}: \HD6, \AD8, \RD6, Schießen-1

		Die Würfe werden summiert. Der beste Gegner wird auf eine 15 kommen, die Robin überbieten muss.

		\subsection{Szene IV: Plan B}

		Geht beim Turnier etwas schief, lässt Robin lachend seine Tarnung fallen, beleidigt den Sheriff, und versucht, über einen provisorisch aufgestellten Holzturm auf die Burgmauer zu kommen. Er ruft den Charakteren \say{Verschafft mir Zeit!} zu und eilt einem noch unschlüssigen Wachtrupp davon (Anzahl SC+1).

		Es kommt zum Kampf. Werden die Wachen überwunden, gibt das Robin genug Zeit, mit Marian -- für alle Anwesenden bestens sichtbar -- entlang der Burgmauer zu fliehen und mit ihr von dort in den Leen zu springen.

	\mysection{Ende gut, alles gut?}

		\noindent
		Das Szenario endet erfolgreich, wenn Robin mit seiner Geliebten Nottingham verlässt -- als Gewinner oder mit ihr fliehend. Werden Charaktere im Szenario überwunden, landen sie in Gefangenschaft im Kerker von Nottingham.
}
